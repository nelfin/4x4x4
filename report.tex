\documentclass[10pt]{article}
\usepackage{times}
\usepackage[fleqn]{amsmath}
\usepackage{amssymb}
\usepackage{anysize}
\usepackage{multicol}
%\marginsize{left}{right}{top}{bottom}:
\marginsize{2.5cm}{1.5cm}{1cm}{1cm}
\begin{document}


\title{COMP3130 -- Group Project in Computer Science\\ Warm-up Project -- 4$\times$4$\times$4 TicTacToe Agent}
\date{March 23, 2012}
\author{Andrew Haigh -- u4667010;\\ Timothy Cosgrove -- u4843619;\\ Joshua Nelson -- u4850020}
\maketitle

\setlength{\columnsep}{18.0pt}
\begin{multicols}{2}
\subsection*{\emph { \textmd{1. Abstract}}}
\hrule
\vspace{0.4cm}
The purpose of this project was to implement an intelligent agent to play 3
dimensional 4 by 4 by 4 Tic Tac Toe. We chose to implement this in C, with a
pipe interface to python. This allowed us to use python's 3D libraries for visualisation
while utilising the speed of a compiled C program.

\subsection*{\emph { \textmd{2. Solution Overview}}}
\hrule
\vspace{0.4cm}
The core of the agent is implemented with a combination of minimax and $\alpha$-$\beta$ pruning.\\
Board states are stored simply as a 3 dimensional array of chars; but this data is passed
through the program as a struct including extra information such as a heuristic evaluation
of the board. For testing purposes all code is compiled with the -g flag. This allows
us to use the program \texttt{gprof} to analyse running time and total number of function calls
(this is used to optimise and evaluate the effectiveness of $\alpha$-$\beta$ pruning).
Modules:\\\\
\texttt{visualisation.py:}
\begin{itemize}
\item The python module which takes user input and displays the game board using VPython
\item This is the main program, it creates a sub-process (\texttt{worker.c}) which returns
board states to display
\item Contains several visualisation options: press 'g' to switch views and 's' to enable
red-cyan stereoscopic 3D.
\end{itemize}
\texttt{worker.c:}
\begin{itemize}
\item The link module which handles communication between python and C
\item Calls code from \texttt{state\_functions.c} to start minimax and pick the next move
\end{itemize}
\texttt{state\_functions.c:}
\begin{itemize}
\item Contains all core functionality of the AI agent including minimax functions and state
evaluation functions
\item Includes a simple victory function which checks if a player has won with a line nearby to
the most recent move. (comprehensive victory checks are made in \texttt{visualisation.py})
\end{itemize}
\end{multicols} 
\end{document}
