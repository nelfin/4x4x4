\documentclass[10pt]{article}
\usepackage{times}
\usepackage[fleqn]{amsmath}
\usepackage{amssymb}
\usepackage{anysize}
\usepackage{multicol}
\usepackage{graphicx}
%\marginsize{left}{right}{top}{bottom}:
\marginsize{2.5cm}{1.5cm}{1cm}{1cm}
\begin{document}


\title{COMP3130 -- Group Project in Computer Science\\ Warm-up Project -- 4$\times$4$\times$4 TicTacToe Agent}
\date{March 24, 2012}
\author{Andrew Haigh -- u4667010;\\ Timothy Cosgrove -- u4843619;\\ Joshua Nelson -- u4850020}
\maketitle

\setlength{\columnsep}{18.0pt}
\begin{multicols}{2}
\subsection*{\emph { \textmd{1. Abstract}}}
\hrule
\vspace{0.4cm}
The purpose of this project was to implement an intelligent agent to play 3
dimensional 4 by 4 by 4 Tic Tac Toe. We chose to implement this in C, with a
pipe interface to python. This allowed us to use python's 3D libraries for visualisation
while utilising the speed of a compiled C program.

\subsection*{\emph { \textmd{2. Solution Overview}}}
\hrule
\vspace{0.4cm}
The core of the agent is implemented with a combination of minimax and $\alpha$-$\beta$ pruning.\\
Board states are stored simply as a 3 dimensional array of chars; but this data is passed
through the program as a struct including extra information such as a heuristic evaluation
of the state. Once the minimax program reaches a fixed cut-off depth we select a state based on
this heuristic. For testing purposes all code is compiled with the -g flag. This allows
us to use the program \texttt{gprof} to analyse running time and total number of function calls
(this is used to optimise and evaluate the effectiveness of $\alpha$-$\beta$ pruning).\\\\
Overview of Modules:\\
\texttt{visualisation.py:}
\begin{itemize}
\item The python module which takes user input and displays the game board using VPython
\item This is the main program, it creates a sub-process (\texttt{worker.c}) which returns
board states to display
\item Contains several visualisation options: press 'g' to switch views and 's' to enable
red-cyan stereoscopic 3D.
\begin{center}
\includegraphics[width=5.3cm]{vis.jpg}\\
{\small Fig 1: Example of game visualisation}
\columnbreak
\end{center}
\end{itemize}
\texttt{worker.c:}
\begin{itemize}
\item The link module which handles communication between python and C
\item Calls code from \texttt{state\_functions.c} to start minimax and pick the next move
\end{itemize}
\texttt{state\_functions.c:}
\begin{itemize}
\item Contains all core functionality of the AI agent including minimax functions and state
evaluation functions
\item Includes a simple victory function which checks if a player has won with a line nearby to
the most recent move. (comprehensive victory checks are made in \texttt{visualisation.py})
\end{itemize}

\subsection*{\emph { \textmd{3. State Space for 4$\times$4$\times$4}}}
\hrule
\vspace{0.4cm}

In the original scope of this project we had intended to examine the entire state space
before making a move. However, the state space is in fact exceedingly large. Considering all
possible states we have $3^{64}$, but many of these (far more than half) are illegal or unreachable.
A mathematical paper by Oren Patashnik$^{[1]}$ points out that after 18 moves (under perfect play by player 1)
the game will be over or in a state where every move is forced. With this in mind we can 
consider a more reasonable upper bound on the states we must examine:
 \[\binom{64}{18} = 3601688791018080\]
If we assume we can examine 1 state per clock cycle (a completely and utterly unreasonable assumption) it would
still take 250 hours to search this reduced state space on a 4GHz CPU. Minimax with $\alpha$-$\beta$ pruning
and other clever tricks could significantly reduce this figure; however it remains obvious that
searching to depth 18 for every move is unachievable. Thus we set a depth cut-off where we stop
searching and perform a heuristic state evaluation.\\\\
Due to this inevitable cut-off it also became impossible to determine if either
player has a winning strategy. It is however interesting to note that (in the paper
mentioned above) Patashnik determined that the first player can always win with perfect play. 

\subsection*{\emph { \textmd{4. Heuristics and Cut-offs}}}
\hrule
\vspace{0.4cm}
briefly explain heuristics and cut off decisions\\



\subsection*{\emph { \textmd{5. Pure Minimax vs. $\alpha$-$\beta$ pruning}}}
\hrule
\vspace{0.4cm}
Some tests were run using the Unix tool \texttt{gprof} to determine the effectiveness
of heuristics and $\alpha$-$\beta$ pruning. By counting the number of calls
to our state-evaluation function we get the total number of examined leaf nodes.
Furthermore \texttt{gprof} returns a summary of how much time (in seconds) is spent running
each function and the whole program.\\
The tables below summarise \texttt{gprof} results when the agent is making it's first move;
run at various cut-off depths:\\\\
Total Leaf Nodes Examined:\\\\
\begin{tabular}{ | l | r r r | }
  \hline                        
  Depth & Minimax & $\alpha$-$\beta$ & $\alpha$-$\beta$ + Heuristics \\ \hline
  3 & 249984 & 8026 & ??? \\
  4 & 15249024 & 256936 & ??? \\
  5 & 914941440 & 622961 & ??? \\
  6 & 53981544960 & 16771844 & ??? \\
  \hline  
\end{tabular}\\\\
Total Time Taken (Seconds):\\\\
\begin{tabular}{ | l | c c c | }
  \hline                        
  Depth & Minimax & $\alpha$-$\beta$ & $\alpha$-$\beta$ + Heuristics \\ \hline
  3 & 11.7 & 0.23 & ??? \\
  4 & ??? & ??? & ??? \\
  5 & ??? & ??? & ??? \\
  \hline  
\end{tabular}\\\\
The benefits of $\alpha$-$\beta$ pruning are obvious and indisputable, often cutting
the number of nodes and total time by a factor of over 1500.\\
However the benefits of our heuristic function are less obvious. We notice that
a great deal less leaf nodes are examined; but the improvement in running time is
negligible. This is due to the fact that we must constantly evaluate states for their
potential value; taking precious CPU time.

\subsection*{\emph { \textmd{6. Summary and Reflection}}}
\hrule
\vspace{0.4cm}

A Summary. Also discuss how this will prepare us for reversi (language choice,
teamwork, gitHub)


\subsection*{\emph { \textmd{7. Bibliography}}}
\hrule
\vspace{0.4cm}
Pfft what am I; an Arts student? I don't know how to make a bibliography.
Man I should get some sleep.
\end{multicols} 
\end{document}
